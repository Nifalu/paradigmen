
%% GENERAL DEFINITIONS
\documentclass{article} %% Determines the general format.
\usepackage{a4wide} %% paper size: A4.
\usepackage[utf8]{inputenc} %% If problem with special characters: "latin1"
\usepackage[T1]{fontenc} %% Format of the resulting PDF file.
\usepackage{fancyhdr} %% Package to create a header on each page.
\usepackage{lastpage} %% Used for "Page X of Y" in the header.
\usepackage{enumerate} %% Used to change the style of enumerations (see below).
\usepackage{amssymb} %% Definitions for math symbols.
\usepackage{amsmath} %% Definitions for math symbols.
\usepackage{amsthm} %% === Definition for proof paragraph ===
\usepackage{color}
\usepackage{comment}
\usepackage{hyperref}

%% NICE ADDITIONS
% \usepackage{parskip} %% uses empty lines for paragraphs instead of \
% \usepackage{MnSymbol,wasysym} % smileys: \smiley{} \frownie{} \blacksmiley{}

%% Remove initial indentation of first line of paragraphs
\setlength{\parindent}{0pt}

%% GRAPHS
\usepackage{tikz}
\usetikzlibrary{automata}
%% Tikz library for nicer arrow heads
\usetikzlibrary{calc,trees,positioning,arrows,fit,shapes,calc} 

%% CODE LISTINGS
\usepackage{listings}

\lstset{
    breaklines=true,
    inputpath=code, 
    language=C++, 
    basicstyle=\fontencoding{T1}\small\fontfamily{lmtt}\fontseries{m}\selectfont, 
    keepspaces=true, 
    numbers=left, 
    stepnumber=1, 
    frame=shadowbox, 
    rulesepcolor=\color{gray}, 
    commentstyle=\color{dkgreen}, 
    keywordstyle=\color{blue},
    showstringspaces=false}

%% DEFINITIONS
\definecolor{dkgreen}{rgb}{0,0.6,0}
\definecolor{gray}{rgb}{0.5,0.5,0.5}
\definecolor{mauve}{rgb}{0.58,0,0.82}

\newtheorem{statement}{Statement}

%% PATHS
\graphicspath{{images/}} %% adds graphics path

%%%%%%%%%%%%%%%%%%%%%%%%%%%%%%%%%%%%%%%%%%%%%%%%%%%%%%%%%%%%%%%%%%%%%%%%%%%%%%%%
%%%%%%%%%%%%%%%%%%%%%%%%%%%%%%%%%%%% HEADER %%%%%%%%%%%%%%%%%%%%%%%%%%%%%%%%%%%%
%%%%%%%%%%%%%%%%%%%%%%%%%%%%%%%%%%%%%%%%%%%%%%%%%%%%%%%%%%%%%%%%%%%%%%%%%%%%%%%%
%% Left side of header
\lhead{\course\\\semester\\Exercise Sheet \homeworkNumber}
%% Right side of header
\rhead{\authorname\\Page \thepage\ of \pageref{LastPage}}
%% Height of header
\usepackage[headheight=36pt]{geometry}
%% Page style that uses the header
\pagestyle{fancy}
%% add title to header
%% \chead{\huge{\textbf{Blatt \homeworkNumber}}}

%%%%%%%%%%%%%%%%%%%%%%%%%%%%%%%%%%%%%%%%%%%%%%%%%%%%%%%%%%%%%%%%%%%%%%%%%%%%%%%%
%%%%%%%%%%%%%%%%%%%%%%%%%%%%%%%%%%% METADATA %%%%%%%%%%%%%%%%%%%%%%%%%%%%%%%%%%%
%%%%%%%%%%%%%%%%%%%%%%%%%%%%%%%%%%%%%%%%%%%%%%%%%%%%%%%%%%%%%%%%%%%%%%%%%%%%%%%%
\newcommand{\authorname}{Nico Bachmann} % TODO: add your names
\newcommand{\semester}{Spring Term 2023}
\newcommand{\course}{Paradigmen und Konzepte von Programmiersprachen}
\newcommand{\homeworkNumber}{3} % TODO: add current sheet nr


\begin{document}
%%%%%%%%%%%%%%%%%%%%%%%%%%%%%%%%%%%%%%%%%%%%%%%%%%%%%%%%%%%%%%%%%%%%%%%%%%%%%%%%
\section*{Exercise \homeworkNumber.1 Bite-sized Haskell Tasks}
\begin{enumerate}[a)]
\item \verb|Swap the values of the first and last element in a list|\\
\lstset{language=Haskell}
\lstinputlisting[linerange={2-5}, firstnumber=2]{main.hs}

First check for the case that the list is empty, if so, return it. Then check the case if the list contains only a single item, if so, again, return it. Only in the third case, when the list contains 2 or more elements, swap the first and last.\\
\verb|last xs| = last element of xs\\
\verb|tail (init xs)| = all elements of xs without the first and last\\
\verb|head xs| = first element of xs \\

\item \verb|Return, as a Boolean, whether two consecutive elements in a list are the same|\\
\lstset{language=Haskell}
\lstinputlisting[linerange={8-11}, firstnumber=8]{main.hs}

Same principle as in a). In the third case however we check if the first two elements are equal and if not, make a recursive call with the second and the third item. So we iterate recursively over the list until we find two equal elements next to ex
\end{enumerate}
\clearpage


%%%%%%%%%%%%%%%%%%%%%%%%%%%%%%%%%%%%%%%%%%%%%%%%%%%%%%%%%%%%%%%%%%%%%%%%%%%%%%%
\section*{Exercise \homeworkNumber.2 Classes and Inheritance}


\clearpage
%%%%%%%%%%%%%%%%%%%%%%%%%%%%%%%%%%%%%%%%%%%%%%%%%%%%%%%%%%%%%%%%%%%%%%%%%%%%%%%%
\section*{Exercise \homeworkNumber.3 Operator Overloading}

\clearpage

%%%%%%%%%%%%%%%%%%%%%%%%%%%%%%%%%%%%%%%%%%%%%%%%%%%%%%%%%%%%%%%%%%%%%%%%%%%%%%%%
\section*{Exercise \homeworkNumber.4 Templates}

%\lstset{language=C++}
%\lstinputlisting{q4.cpp}
\clearpage

%%%%%%%%%%%%%%%%%%%%%%%%%%%%%%%%%%%%%%%%%%%%%%%%%%%%%%%%%%%%%%%%%%%%%%%%%%%%%%%%
\section*{Exercise 5 and 6 Tic Tac Toe, The Ultimate Game}


\end{document}